\documentclass[11pt]{article}

\usepackage[nohide,twocolumn]{ulecnot}

\pagestyle{fancy}
\lhead{COGS 501 -- Formal Languages and Linguistics}
\chead{General Introduction}
\rhead{Last updated \it \today}
\lfoot{Umut \"Ozge}
\cfoot{}
\rfoot{Page \thepage/\pageref{LastPage}}
\setlength{\headheight}{13.6pt}

\usepackage[T1]{fontenc}
\usepackage[utf8]{inputenc}
\usepackage{linguex}
	\renewcommand{\refdash}{}

\begin{document}
\section*{General introduction: Why this course?}

\begin{itemize}
\item Why do we start the program with linguistics and formal languages?
	
\item Some say ``language is a window into mind''. What do they mean by that? 
\end{itemize}

\begin{itemize}
\item The central notion in the cognitive science of language -- and possibly
in other domains -- is \uterm{grammar}. We will repeatedly deal with the concept
throughout the term.
\item For a language $L$, a grammar is an explicit specification of the principles governing the
formation of words, phrases and sentences in $L$, and their interpretation.
\item A satisfactory grammar for Turkish not only should realize that 
\xxref{akutusu}{a} is, but  \xxref{akutusu}{a} isn't, a well-formed noun phrase
with the main emphasis is on the first word, rather than the second,
but also that the entity expressed by \xxref{akutusu}{a} is a sort of box rather
than a shoe -- knowing what words mean is not enough.


\ex.\label{akutusu}
\a. ayakkabı kutusu
\b. *kutusu ayakkabı\\
	Intended:`shoe box'.

\item Genuine philosophical and scientific inquiry starts at the moment you
begin to discover that something that you thought was completely ordinary and
expected, is not so. 

\item How do speakers of Turkish know(!) the facts in \xref{akutusu}?

\item Could general cognitive mechanisms like memorization or stimulus-response conditioning be adequate for this? Let's think for a while.
\end{itemize}

\begin{itemize}
\item Here are some facts about Turkish that you may not be aware of before. 

\item[\bf A.] If you look at it closely, you realize that usually words are
stressed on their final syllable. However if the word is a place name, then the
stress shifts to an earlier syllable. 

\item[] Can it be the case that I stress \emph{Ayranc\i{} dolmu\c su}  in the way
I do because that's the way I heard it?

\item[\bf B.] Another example, assume a learner of Turkish says s/he is puzzled about how to put an accusative case (Tr. ``ismin `i' hali'') in Turkish?   How would you describe the rules? 

\item[\bf C.] Turkish compounds:
	
\item[] Type 1: dolap + kilit = dolap kilidi\\
		Type 2: trafik + ceza = trafik cezası

		Try possesive suffixes on these, how do they differ? Why?	

\item[\bf D.] Restrictions on Turkish expressions \emph{hiç} and \emph{bile}.   

\item[] First some English:

\ex. \a. John didn't drink any water.
\b. Did John drink any water?
\b. If John drank any water, he must be in trouble.
\b. *John drank any water.

\item[] Now Turkish:

\ex. 
\a. Ahmet hiç su içmedi.
\b. Ahmet hiç su içti mi?
\b. *Ahmet hiç su içti.
\b. *Ahmet hic su ictiyse, hapi yuttu demektir.

\ex. 
\a. Ahmet bir damla su bile içmedi.
\b. Ahmet bir damla su bile içti.
\b. *Ahmet bir damla su bile içti mi?
\b. Ahmet bir damla su bile ictiyse, hapı yuttu demektir.

\ex.
\a. Ahmet bir damla bile su içmedi.
\b. *Ahmet bir damla bile su içti.
\b. ?Ahmet bir damla bile su icti mi?
\b. Ahmet bir damla bile su ictiyse, hapı yuttu demektir.

\item[\bf E.] Under what circumstances is the following sentence true? 

\ex. Her öğrenci bir romancının yazdığı her romanı okudu.


\item[\bf E.] Who can the pronouns refer to 
in these sentences?  What is the rule?

\ex. 
\a. Ahmet [Ayşe'nin onu sevdiğini] biliyor.
\b. Ahmet [onun Ayşe'yi sevdiğini] biliyor.
%\b. Mehmet Ahmeti [akşam ona gıtmeye] ikna etti.
%\b. Mehmet Ahmeti [onun bu işi başaramayacağına] ikna etti.
%\b. Mehmet Ahmeti [o eve varmadan] aradi.
%\b. Mehmet Ahmeti [mektup ona ula\c smadan] arad\i.

\item I think we are at a point to justifiably claim this:

\ex. {\bf Proposition:} 
Language operates via rules and principles not accessible to
consciousness.


\item When speaking of rules we do not mean rules on ``proper'', ``good'',
``effective'' speaking (\uterm{prescriptive} rules), we mean rules on what is
possible to occur in nature (\uterm{descriptive} rules)-- compare with the rules of other sciences.  



\ex. {\bf Observation:}
Language is creative; everyday we hear many sentences that we never heard
before, and this will continue to be so. 


\item Memorization alone does not work because\ldots

\item Stimulus-response conditioning alone does not work because\ldots


\item How do you think such a system of rules and restrictions is mastered?
The technical term is \uterm{language acquisition}.

\item Some crucial observations on language:
	\begin{itemize}
\item[i.] Languages are acquired by kids without any training on the
grammar of their native language, starting at very early ages;
 who are not yet capable of adult-like reasoning; and even by those
non-typical in their cognitive development, for instance those lacking
communicative capacities like ``reading'' others' minds, attending to the same
thing with others, interpreting emotional states and gestures, and so on.

\item[ii.] Language acquisition follows a similar course
regardless of where the kid is born and what natural language she acquires. 

\item[iii.] There is a \uterm{critical period} of language acquisition (roughly
the first 8 years) during which language ``matures'' in the mind/brain of
the child, and beyond which native-level competence in another language is
virtually impossible.\footnote{Beyond this period, we switch to the phrase
``language learning'' (in contrast to ``acquisition''). 
 If no language is acquired in
the critical period, even a basic level of competence is unattainable.}

\item[iii.] Language is uniquely human; no successful attempts to
observe anything similar in other species.
\end{itemize}

\item These facts make cognitive scientists think that there
are \uterm{universals} of human language, a core which is common to all natural
languages acquirable by human beings.
Or from another angle, the capacity to acquire and use language is a defining
characteristic of our species.

\item One influential hypothesis is that humans are born with a genetically
determined capacity for language residing in their brains, which guides the
language acquisition process so that they end up with an abstract system of
rules -- a grammar -- determining how sound is paired with meaning.\footnote{We
abstract away from written and signed forms of language.}

\item It has many historical precedents but the idea is first systematized into
a research program by Noam Chomsky:\footnote{Adapted from p.\ 3 of Chomsky N.
(1988).  \emph{Language and Problems of Knowledge, the Managua Lectures}. The
MIT Press: Cambridge, MA.}

\item[]  

\ex.\label{chomskyquestions}
\a. What is the system of knowledge? What is in the mind/brain of the speaker of
English or Spanish or Japanese?
\b. How does this system of knowledge arise in the mind/brain?
\b. How is this knowledge put to use in speech?
\b. What are the physical mechanisms that serve as the material basis of this
system of knowledge and for the use of this knowledge. 


\item Throughout the term we will train ourselves in some methods and attempts
to address the question \xxref{chomskyquestions}{a}.

\item Our focus is what makes human language unique among other similar
systems of representation and communication. 

\item We will see that there are types (or families) of languages grouped
according to the complexity of their structure.  We will see that a case can be
made about the location of human languages in this hierarchy of types.

\item We will see that this might in turn serve a key to what makes human
language capacity what it is.

\end{itemize}

\end{document}

Because the girl that the teacher of the class admired didn't call her mother
was concerned.
