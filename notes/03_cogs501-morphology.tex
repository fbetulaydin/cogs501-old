\documentclass[11pt]{article}
\usepackage{tipa}
\usepackage[nohide,twocolumn]{ulecnot}

\pagestyle{fancy}
\lhead{COGS 501 -- Formal Languages and Linguistics}
\chead{Morphology}
\rhead{Spring 2017}
\lfoot{Umut \"Ozge -- \href{mailto:umozge@metu.edu.tr}{\nolinkurl{umozge@metu.edu.tr}}}
\cfoot{\today}
\rfoot{Page \thepage/\pageref{LastPage}}
\setlength{\headheight}{13.6pt}


% \usepackage[screen,article]{pdfscreen}
% \usepackage[display]{texpower}

\usepackage[T1]{fontenc}
\usepackage[utf8]{inputenc}
\usepackage{linguex}
	\renewcommand{\refdash}{}

\newcommand{\morphrule}[3]{{\it#1}\hspace{10pt}\sysm{\imp}\hspace{10pt}{\it#2}\hspace{10pt}/\hspace{10pt}{\it#3}}


\begin{document}


\section{Introduction\protect\footnote{The material up to section \ref{morphop}
is largely adapted from Payne (2006).}}

\begin{itemize}
\item Morphology deals with 
\ezimeti{
\item word form (sound shapes)
\item word structure (abstracting from the shape)
\item word generation (building complex structures)
}
\item It aims to define notions ``morpheme'', ``word'', ``inflection''
\item But why a cognitive scientist should care?
\item Because human beings seem to have intuitions (read as implicit knowledge)
about these concepts.
\item How do we know that they do?
\item It is revealed in behavior and learning. 
\end{itemize}


\section{Morphemes, morphs, allomorphy}

\ezimeti{
\item What comes to your mind when you hear ``word structure'', or just
``structure''?
\item Let's start with something basic and intuitive: part-whole structure of a
word.  

\item {\bf Morpheme:} smallest linguistic unit with a grammatical function
\sysm{\simeq} smallest meaningful part. For the notion of morpheme, we take
meaning to be equivalent to function. By ``smallest'', we mean a unit that is
not decomposable further.  

\item Another definition would be ``sound-meaning pairing''. In some cases this
proves problematic, we will come to them below.



\begin{uexercise}[Aztec]
List the morphemes:

\begin{tabular}{llllll}
a. &ikalwewe &`his big house' &i. &petatc$\cdot$in&`little mat'\\
b. &ikalsosol &`his old house' &j. &ikalmeh &`his houses'\\
c. &ikalc$\cdot$in &`his little house' &k. &komitmeh &`cooking-pots'\\
d. &komitwewe &`big cooking-pot' &l. &petatmeh &`mats'\\
e. &komitsosol &`old cooking-pot' &m. &ko$\cdot$yamec$\cdot$in &`little pig'\\
f. &komitc$\cdot$in &`little cooking-pot' &n. &ko$\cdot$yamewewe &`big male pig'\\
g. &petahwewe & `big mat' &o. &ko$\cdot$yameilama&`big female pig'\\
h. &petatsosol & `old mat' &p. &ko$\cdot$yamemeh&`pigs'
\end{tabular}
\end{uexercise}


\item Morpheme classification:

\item[] {\bf Free} morphemes can stand alone, while {\bf bound} morphemes always
need some other linguistic material to realize a complete(??) expression.


\item[] Another typing:

\ezimeti{
\item Root (bound or free)
\item Affix (bound)
\item Clitic (bound)
}

\item Root is the semantically main part of a word, which cannot be further
divided into morphemes. Examples?

\item Sometimes linguists talk of \uterm{stems} as well. The stem root
distinction is subtle and sometimes confusing. 

\item Affixes are added to a root or stem to form new words. 

\item Clitics are bound morphemes that differ from affixes by being related to a
structure larger than those at word-level. E.g.\ Turkish \emph{de/da} are clitics. 

}

\subsection{Some complications with the notion ``morpheme''}

\ezimeti{
\item Non-segmental morphemes: the morpheme does not correspond to a designated
sequence of sounds (segment), but is more abstract.

\item[] English past tense:

\begin{tabular}{ll}
run & ran \\
speak & spoke\\
eat & ate\\
read & read
\end{tabular}


\item[] Noun to verb morphology:


\begin{tabular}{l|l}
Noun & Verb \\ \hline
breath & breathe\\
cloth & clothe\\
house & house
\end{tabular}

where the voicless fricative endings ([\textipa{T,s}]) turn to voiced fricatives
([\textipa{D,z}])

\item Zero morphemes:
\item[] Turkish:

\begin{tabular}{ll}
gidiyor-um & gidiyor-$\emptyset$ \\
going-1sg & going-3sg
\end{tabular}

\item[] English:

\begin{tabular}{l}
a fish\\
ten fish-$\emptyset$
\end{tabular}



\item Two lessons:
\item[] Some morphological operations need to be thought as processes.
\item[] At an abstract level a morpheme defines a contrast.





\begin{uexercise}[German plural] 
What indicates plurality in German?

\begin{tabular}{llll}
V\"ater & `fathers' & Auge & `eye'\\
Kinder & `children' & Adler & `eagle'\\
Pferd & `horse' & Kind & `child'\\
M\"anner & `men' & Augen & `eyes'\\
Vater & `father' & Kuh & `cow'\\
Mann & `man' & Frauen & `women'\\
Adler & `eagles' & Auto & `car'\\
K\"uhe & `cows' & Autos & `cars'\\
Pferde & `horses' & Frau & `woman'
\end{tabular}
\end{uexercise}


\item {\bf Morph:} individual \uterm{tokens} of a morpheme.

\item {\bf Allomorphs:} The set of morphs of a morpheme. Two morphs from this
set stand in an allomorphy relation.

\item Example: English plural allomorphs?



\item Note that in deciding on morphemehood, function (meaning) prevails over form. Compare \emph{quickly} \versus
\emph{lovely}.

\begin{uexercise}
Think of cases from Turkish where we need the abstraction of allomorphy.
\end{uexercise}
}

\section{Inflection \versus\ derivation}
\ezimeti{
\item Inflections express grammatical or functional categories -- this will make
more sense when we come to syntax. 
\item[] therefore they are general. E.g.\ In a given language if a verb has
tense every verb has tense; if a noun has case, every noun has case.
\item[] Inflections are relevant for syntax.

\item Derivations form new words.  
\item[] they are more restricted: \emph{build-ing}, \emph{*see-ing},
\emph{view-ing}\ldots
}

\section{Main morphological processes}
\ezimeti{
\item[] {\bf Affixation:}

\ezimeti{
\item Prefix: \emph{anti-dis-establishment}

\item Suffix: \emph{antidisestablish-ment-ari-an-ism}

\item Infix:

\item[] Tagalog \emph{-um-} makes an agent out of a verb:

\begin{tabular}{lll}
sulat& s-um-ulat & `one who wrote'\\
gradwet & gr-um-adwet & `one who graduated'
\end{tabular}

\item Circumfix:

\item[] Indonesian \emph{ke-\ldots-an} makes nouns from adjectives:

\begin{tabular}{llll}
besar & `big' & ke-besar-an & `bigness, greatness' 
\end{tabular}

\item[] Chukchee  \emph{a-\ldots-ke} expresses negation:

\begin{tabular}{llll}
jatjol & `fox' & a-jatjol-ka & `without a fox' \\
cakett  & `sister' & a-cakett\textipa{@}-ke & `without a sister'
\end{tabular}

}


\item[] {\bf Stem modification:}

Shape change without an affix.
Ex.: \emph{ring}, \emph{rang}.


\item[] {\bf Autosegmental variation:}

What is ``autosegmental''? Processes that refer to structures beyond single
sounds are autosegmental processes or properties. For instance where the main stress (Tr. \emph{vurgu}) refers to
syllables (and their organization) rather than individual sounds; there is no
rule like ``stress the vowels \emph{a} and \emph{e}, but not \emph{i}, \emph{o},
and so on''.

English stress: 

\emph{conv\'ert} $\rightarrow$ \emph{c\'onvert} 

\emph{perm\'it} $\rightarrow$ \emph{p\'ermit}


\item[] {\bf Reduplication:}

\item[] Indonesian plural is made by duplicating the root: \emph{anak} `child',
\emph{anakanak} `children'.

More interestingly, Ilakona, an Austronesian language, duplicates the first
syllable only for plural: \emph{ulo} `head', \emph{ululo} `heads'. 


	

\item[] {\bf Non-concatenative morphology:}

In Semitic languages like Arabic and Hebrew.

\begin{tabular}{lll}
ktb & Root & no meaning\\
k\textipa{@}tob & imperative & `write!'\\
\textipa{kAtob} & infinitive & `to write'\\
\textipa{kotEb} & present participle &  `writing'\\
\textipa{kAtub} & past participle & `written'\\
\textipa{kAtab} & perfective & `wrote'
\end{tabular}

\item[] {\bf Subtractive morphology:}

Murle (East Africa) plurals are made by removing the final consonant:
\emph{nyoon} `lamb', \emph{nyoo} `lambs', \emph{onyiit} `rib', \emph{onyii}
`ribs'.

Don't confuse with languages that mark the singular rather than the plural:

Arbore (Ethiopia) is such a language: \emph{tiisin} `a maize cob', \emph{tiise}
`maize cobs', \emph{nebelin} `an ostrich', \emph{nebel} `ostriches'.


\item[] {\bf Compounding:}

black + board = blackboard


}

\section{Morphotactics (morphology-syntax/semantics\protect\footnote{It will
become clear in the coming weeks why we write ``syntax/semantics''.} interaction)}
\ezimeti{
\item  Morphotactics is the part of the grammar that regulates the order of
morphemes.
For instance in Turkish, the order causative >  passive > aspect > person
holds, e.g.\ \emph{kand\i{r\i{l}m\i{\c s\i{z}}}}.
\item Morphotactics involves notions from both syntax and semantics. For
instance, the reason why passive cannot ``apply'' before causative has a
syntactico-semantic explanation. 
}
\section{Rules and representations in morphology}
\ezimeti{
\item Let's take the plural marking in English (N = noun, pl = plural, sg =
singular):

\ex. N$_{sg}$ + -\emph{s} = N$_{pl}$  

\item We can represent the process in an input/output format:

\ex. \emph{Name of the process}: \emph{input} \sysm{\imp} \emph{output}

\ex. Plural: N \sysm{\imp} N + -\emph{s}

\item Now some more abstraction. Take the following data from Arabic.

\ex.
\begin{tabular}[t]{llllll}
 &Root: & slm & &Root: & ktb\\
a. & muslim & `person of peace' & c. & muktib & `literate person'\\
b. & salima & `he was safe' & d. & katiba & `he was reading'
\end{tabular}

\item[] On the surface the rule appears as:

\ex. 
\a. Person nominalization: slm \sysm{\imp} muslim
\b. Person nominalization: ktb \sysm{\imp} muktib

\item[] It is not hard to see that there is room for being more general:
\ex.\label{arabicpers} Person nominalization: C$_1$C$_2$C$_3$ \sysm{\imp}
\emph{mu}C$_1$C$_2$\emph{i}C$_3$

\item Observe that we have a type of rule which is able to take apart the consonants in
an Arabic root and store them in variables. Once this is done, with similar
rules you can perform any operation of shuffling, insertion and/or deletion on
the input. Also observe that a rule capable of breaking into
parts a string of an arbitrary length (not just 3) would need to be more
``intelligent'' than what we have here. We will see such rules in coming weeks.  

\item By the way, can you see an important shortcoming of these rules? 

\item Some crucial information that might be needed by another rule that would
act on the result of the above rules is missing.

\item Usually morphology is not a business done only for its own sake. Many
properties that are decided by morphological processes are used by the rules of
syntax; just like morphology sometimes uses outputs of phonological processes. Therefore
in a more thoroughly conceived version of \xref{arabicpers}, we would have the information that the
output is a noun (since the rule is nominalization), so that the upcoming 
syntactic processes can make a proper use of it. 

}
\newpage
\section{Morphophonemics (morphology-phonology interaction) (optional material)}
\label{morphop}

\ezimeti{ 
\item As we saw while discussing the notion of allomorphy the
particular form (the morph) of a morpheme may depend on the particular
\uterm{environment} of the morph. Rules governing such processes fall within
\uterm{morphophonemics} (or \uterm{morphophonology}.

\item For a process to fall under morphophonemics it must  refer
to morphological units like morpheme, stem, affix, and so on, when
specifying the process. Otherwise processes involving only sound (or form) change
without reference to morphology falls in the realm of \uterm{phonology}.
Examples?

\item Let's start with Turkish perfective. The allomorphs are:\\
$\{${\it -d\i,-di,-du,-d\"u,\mbox{-t\i},\mbox{-ti},-tu,-t\"u}$\}$.

\item[] or\\ \sysm{\crbr{\text{-}\alpha\beta\,|\, \alpha \in \crbr{\text{\it d,t}}, \beta \in
\crbr{\text{\it \i,i,u,\"u}}}}


\item In what follows, we will concentrate on consonants and leave aside vowels
and vowel harmony. So let $H$ stand for a vowel which is picked from  $\crbr{\text{\it \i,i,u,\"u}}$ according
to vowel harmony. Also $C$ stands for all the consonants, and $V$, for vowels. 
By convention, these set names can stand both for the sets and an element picked
from the sets. The particular sense we are using these symbols will be clear
from the context. 

\item[] Given these conventions, the allomorphs of Turkish perfective are \{{\it -dH,
-tH}\}. Now what is the rule that governs whether we have \emph{t} or \emph{d}?


\item We are faced with the tasks of specifying:
\ezimeti{
\item[A.] which consonant appears in which environments.
\item[B.] what will be the input to the morphophonemic rule(s).
}

\item Now, let's make an assumption which has the potential of simplifying
things. Let's assume that  the following rule attaches an abstract morpheme of
perfective to a verbal stem. If we can maintain this, our sound changing rules
do not need to ``know'' about the details of the morphemes. Abstraction for the
consonants can be accomplished by defining $D$ = \{\emph{t},\emph{d}\}, like we
did for vowels -- but this time we will specify what $D$ will become
ourselves.\footnote{Here is an idealization. For the sake of simplicity we will
treat expressions like \emph{tuhaft\i} comprising of a verb ending with the
sound `f' and the perfective marker. A more thorough analysis is to have the
adjective \emph{tuhaf} followed by the copula \emph{i} and the perfective
marker. Turkish morphophonemics gives the final form \emph{tuhaft\i} to the
composition of all these.}

\ex. Perfective: Verb \sysm{\imp} Verb + \emph{DH}


\item Now we can devise a rule that further specifies $D$ as \emph{t} or
\emph{d} according to the specific environment. Here is a tentative rule for
that -- environments are shown by giving a \uterm{context} after the `/'.  

\ex. 
\a. \morphrule{DH}{tH}{$\{$p,\c c,t,k,s,\c s,f,h$\}$\cntx}
\b. \morphrule{DH}{dH}{$V$ {\rm or} $C -$ $\{$p,\c c,t,k,s,\c s,f,h$\}$\cntx}


\item We can take the $H$ repeated in the input specification to the environment
specification:

\ex.\label{tenta} 
\a. \morphrule{D}{t}{$\{$p,\c c,t,k,s,\c s,f,h$\}$\cntx$H$}
\b. \morphrule{D}{d}{$V$ {\rm or} $C -$ $\{$p,\c c,t,k,s,\c s,f,h$\}$\cntx$H$}

\item We can simplify this rule by assuming that one of \emph{d} and \emph{t} is the
default case, and it is converted to the other in certain environments. What should
be the basis of which form to take as the basic? One way to tackle such issues,
which is quite common in science, is to make an assumption and look at the 
consequences of this assumption. Let's take \emph{t} as default and see what happens. Our rule
becomes:

\ex.\label{tasdefault} \morphrule{t}{d}{$V$ {\rm or} $C -$ $\{$p,\c c,t,k,s,\c
s,f,h$\}$\cntx$H$}

\item This rule changes all the \emph{t}'s to \emph{d}'s if they are found to be
preceded by a vowel or a consonant from the set specified in the environment
part and followed by $H$, and leaves untouched other \emph{t}'s. Observe that
the effect is the same with that of rule \xref{tenta}, but now we have a simpler
rule.


\item One immediate problem with this rule is its behavior in cases like
\emph{\c sart\i} (as in \emph{\"uyelik \c sart\i}. Our rule would erroneously
convert \emph{\c sart\i} to *\emph{\c sard\i}.\footnote{To be precise, here we
assume that $H$ has not been turned to \emph{\i} yet at the time our rule is
applied. Therefore what happens is: first our rule turns \emph{\c sart$H$} to
\emph{\c sard$H$}, and the vowel harmony rule, which we leave unspecified here,
turns this into \emph{\c sard\i}. This is to say that vowel harmony rules apply
after the rule we are tyring to specify.} This can be avoided by a
simple modification:  we just make our rule aware of morpheme boundaries so that
it does not operate on the \emph{t} in \emph{\c sart\i}, which is now \emph{\c
sart-\i}, and thereby does not match the input specification of our new rule in
\xref{new}.


\ex.\label{new} \morphrule{-t}{-d}{$V$ {\rm or} $C -$ $\{$p,\c c,t,k,s,\c s,f,h$\}$\cntx$H$}


\item However, another problem arises in cases like \emph{karar-t\i},
\emph{\"urper-ti}, and the like. Our rule would erroneously turn these to \emph{karar-d\i} and
\emph{\"urper-di}, which are, though well-formed, entirely different than what
we intend. To remedy this, we can make our rule aware of the distinction between
derivational \versus\ inflectional morpheme boundaries. If we agree that `+'
signifies derivational, and `-' signifies inflectional boundaries, than our rule
would not make the changes above, as the actual representations of these are
\emph{karar+t\i} and \emph{\"urper+ti}.

\item It is crucial to observe the chain of reasoning which lead us from the
decision to take \emph{t} as the default, to the conclusion that our rules need to be
aware of the derivation \versus\ inflection distinction. Now let us
\uterm{backtrack} to an earlier decision point, retract our ``\emph{t} is
default'' assumption and instead take \emph{d} as the default.  Now the rule
becomes:


\ex.\label{dasdefault} \morphrule{-d}{-t}{$\{$p,\c c,t,k,s,\c s,f,h$\}$\cntx$H$}


\item To the best of our knowledge, \xref{dasdefault} does not run into the
problem we had with default \emph{t} in the face of expressions like
\emph{karart\i}. In other words, our conjecture is that there will be no cases
where a \emph{d} right after a morpheme boundary would erroneously be turned to
\emph{t}. Of course the rule would need to be revised upon encountering a
possible counterexample.

\item The above observation, if correct, also frees our hands of the
derivational \versus\ inflectional morpheme distinction. The `-' can now mean
``any morpheme boundary''. Upon closer inspection perhaps we may even drop the
morpheme boundary sign, but we will not pursue this investigation here. 


% \ex.\label{tentb}
% \a. \morphrule{-DH}{-tH}{$\{$p,\c c,t,k,s,\c s,f,h$\}$\cntx}
% \b. \morphrule{-DH}{-dH}{$V$ {\rm or} $C -$ $\{$p,\c c,t,k,s,\c s,f,h$\}$\cntx}
% 
% \item This still leaves some possibility of overgeneration. Can you see how?
% 
% \item In order to avoid this overgeneration, our rules should be made aware of the distinction
% between derivational \versus\ inflectional morphemes. Let's agree that `+'
% designates a derivational, and `-' an inflectional boundary. Our rules stay the
% same, but now their meaning is different.
% 
% \item Now the rules for the perfective in \xref{tentb} would not affect
% `\emph{avun}+\emph{DH}', `\emph{kas\i{}n}+\emph{DH}',
% and so on, as it will ``realize'' that these are meant to be derivations --
% formations of new words -- rather than inflections. I leave as an exercise to
% characterize the rule for the derivations just mentioned.  

\item There is still room for further simplification, this time of a different
character. The simplification involves finding a \uterm{natural class} for the set of
consonants in the environment specification of the rule, something that serves a common
denominator for the consonants in the list, by the use of which we can represent
all of them at once. For that, we need to digress a little. 


\item[] {\bf Very basic phonetics/phonology:}

\item Think of a consonant, say \emph{t} or \emph{b}, as being specified by three
\uterm{features}: (i) where (\uterm{place}) in the vocal tract it is articulated
(alveolar, dental, glottal, and so on), (ii) how (\uterm{manner}) it is
articulated (fricative, plosive, nasal, and so on); and (iii) whether the vocal
chords are involved or not (\uterm{voice}). A particular consonant can be
represented as a set of features.

\item For example, \emph{t} and \emph{d} are both ``alveolar plosives'',
articulated by restricting (hence plosive) the air flow by tongue and alveolar
ridge (hence alveolar). The former is voiceless, while the latter is voiced.
More formally, \emph{t} is \{[place alveolar],[manner plosive],[voice -]\},
while \emph{d} is \{[place alveolar],[manner plosive],[voice +]\}. We adopt
a abbreviating convention, where we omit the words ``place'' and ``manner'' and
write `[+voice]' instead of `[voice +]', etc. 

\item Returning to our track, the set of consonants that call for a \emph{d} to
\emph{t} conversion in
\xref{dasdefault}  is the class of voiceless consonants (of Turkish)([-voice]). If we
designate this set as $C_{\text{[-voice]}}$, then the rule becomes:


\ex.\label{voice} \morphrule{-d}{-t}{$C_{\text{[-voice]}}$\cntx$H$}


% \item An important \emph{type} of simplification involves the choice of the underlying
% form. Let's see what we gain by taking one of the morphs as the underlying form
% and resort to change only when it is needed. Which one would you take as the
% underlying form, \emph{t} or \emph{d}?
% 
% \ex. \morphrule{-dH}{-tH}{$C_{\text{[-voice]}}$\cntx}
% 
% \item Now we have a fairly simple rule -- but isn't there room for more
% generalization?

\item To further generalize our rule, observe the use of suffixes for locative (\emph{-de,-da,-te,-ta}),
adjective forming (\emph{-ken,-kan,-gen,-gan}), occupational (\emph{-cH,-\c{c}H}).
Again the consonants are alternating between voiced and voiceless variants of the
same type. Therefore, why not have rules like the following, entirely leaving the
consideration of ensuing vowels out:

\ex. \morphrule{-d}{-t}{$C_{\text{[-voice]}}$\cntx}

\ex. \morphrule{-g}{-k}{$C_{\text{[-voice]}}$\cntx}

\ex. \morphrule{-c}{-\c c}{$C_{\text{[-voice]}}$\cntx}

\item Would it be possible to collapse these rules into one? Actually same thing
is going on in all of them. What is that?

\item First, let's agree that all the vowels are [+voice] by definition. 

\item[] Second, let's agree to use  variables that can stand for the values of
features, the minus or plus signs for voice, glottal, dental, etc. for place,
and so on. 

\item[] Third, let's use `?' as the value of a feature, whose value we do not
care -- whatever comes in its place is fine for us. 

\item[] Forth, let's define a \emph{meta} feature `[cons. +]' or `[cons. -]',
which is possessed by all consonants.\footnote{One way to define is to introduce
a conditional rule: If a feature set $F$ has at least one feature which can only be
possessed by consonants, add `[cons. +]' to $F$, otherwise add `[cons. -]'.} 

\item[] Now, all this in place, we can state:
{\footnotesize
\ex.\label{varrule} \morphrule{-{\rm \{[+cons.],[?voice]\}} $\cup\, A$}{{\rm -\{[+cons.],[$\alpha$voice]\} $\,\cup\, A$}}{$Z\, \cup$  {\rm \{[$\alpha$voice]\}}\cntx}

}
\item Let's look at what's going on in \xref{varrule} in detail. First our
underlying form, which is the input to our rule, leaves the voice feature of the
consonant \uterm{underspecified}. This is to say that whether the consonant will
be voiced or not is left to \xref{varrule} to decide. Our rule says that the
voice feature of the initial consonant of a morpheme has to be same with the
voice feature of the consonant preceding the morpheme boundary. 

\item This move brings a huge simplification to our morphophonemics, but there
is a complication, namely, the problem with \emph{karart\i}, \emph{\"urperti}
came back. What is more, this time we cannot solve the problem by counting on
the derivation \versus\ inflection distinction, since our generalization about
voice was meant to apply to both cases. At this point one need to decide whether
to keep the generalization in \xref{varrule} or not. If one decides to keep,
then a way of covering the ``exceptional'' case of \emph{karart\i} needs to be
found. Maybe there is ``something'' in that case that makes the morphophonemics
behave differently. The task would be to find that ``something'' and revise the
rules so that they become ``aware'' of the necessary distinction. I leave the
further development of the account as an exercise for the interested.   

\item So far we have seen rules that modify a sound or an underlying form. One
can also devise rules that insert and/or delete -- even shuffle -- sounds. It is
also possible to write rules that are sensitive to syllable boundaries or any
other information. Just don't forget that any extra information you introduce
needs to be justified on the grounds that it is required for a correct and/or
simple account of the phenomenon.

\item When you try other examples, you will realize that in most cases rules
should be ordered. For a very simple instance, rules that delete the `+'s and
`-'s in the input should come last.

\item Finally, there is still room for improvement in the rule schema we have used. We
will come back to it after we have a more formal look at the subject in the next
weeks.

\hrulefill
\begin{uexercise}
\ezimeti{
\item[]
\item[i.]
In our first lecture we observed that when the accusative case marker is
suffixed to a possesive compound (e.g.\ \emph{trafik cezas\i}), the sound
\emph{n} needs to be inserted in between. Give an account of this process. One crucial
question is whether the \emph{n} is part of the accusative marker or the
possessive suffix. 

\item[ii.]\label{trredup}
Give an account of emphatic reduplication in Turkish, e.g.\
\emph{sar\i{}}\sysm{\imp}\emph{sapsar\i}, \emph{ye\c sil}\sysm{\imp}\emph{yemye\c sil}, and so on.
}
\end{uexercise}

\hrulefill
}

\end{document}
